% hungtran
\documentclass[13pt]{beamer}
%
% Choose how your presentation looks.
%
% For more themes, color themes and font themes, see:
% http://deic.uab.es/~iblanes/beamer_gallery/index_by_theme.html
%
\mode<presentation>
{
  \usetheme{CambridgeUS}     % or try Darmstadt, Madrid, Warsaw, ...
  \usecolortheme{beaver} % or try albatross, beaver, crane, ...
  \usefonttheme{default}  % or try serif, structurebold, ...
  \setbeamertemplate{navigation symbols}{}
  \setbeamertemplate{caption}[numbered]
} 

\usepackage[english]{babel}
\usepackage[utf8x]{inputenc}
\usepackage{xcolor}
\usepackage{multicol}
\usepackage{tikz}
\usepackage{tikz-uml}
\tikzumlset{font=\footnotesize\ttfamily, class width=6ex}
\usepackage{hyperref}

\usepackage{listings}
\definecolor{codegreen}{rgb}{0,0.6,0}
\definecolor{codegray}{rgb}{0.5,0.5,0.5}
\definecolor{codepurple}{rgb}{0.58,0,0.82}
\definecolor{backcolour}{rgb}{0.95,0.95,0.92}

\lstdefinestyle{myCustomCppStyle}{
  language=C++,
  numbers=left,
  stepnumber=1,
  numbersep=9pt,
  tabsize=2,
  showspaces=false,
  showstringspaces=false
}

\lstset{basicstyle=\tiny,style=myCustomCppStyle}

\lstdefinestyle{mystyle}{
    backgroundcolor=\color{backcolour},   
    commentstyle=\color{codegreen},
    keywordstyle=\color{magenta},
    numberstyle=\tiny\color{codegray},
    stringstyle=\color{codepurple},
    basicstyle=\ttfamily\footnotesize,
    breakatwhitespace=false,         
    breaklines=true,                 
    captionpos=b,                    
    keepspaces=true,                 
    numbers=left,                    
    numbersep=5pt,                  
    showspaces=false,                
    showstringspaces=false,
    showtabs=false,                  
    tabsize=1
}

\lstset{style=mystyle}

\usepackage{graphicx}
\graphicspath{ {./images/} }

\usepackage{tikz}
\usetikzlibrary{decorations.text}
\usetikzlibrary{shapes.geometric, arrows, positioning, calc, matrix}

\tikzset{
  basic box/.style={
    shape=rectangle, rounded corners, align=center,
    draw=#1, fill=#1!25},
  header node/.style={
    Minimum Width=header nodes,
    font=\strut\Large\ttfamily,
    text depth=+0pt,
    fill=white, draw},
  header/.style={%
    inner ysep=+1.5em,
    append after command={
      \pgfextra{\let\TikZlastnode\tikzlastnode}
      node [header node] (header-\TikZlastnode) at (\TikZlastnode.north) {#1}
      node [span=(\TikZlastnode)(header-\TikZlastnode)] at (fit bounding box) (h-\TikZlastnode) {}
    }
  },
  hv/.style={to path={-|(\tikztotarget)\tikztonodes}},
  vh/.style={to path={|-(\tikztotarget)\tikztonodes}},
  fat blue line/.style={ultra thick, blue}
}

\definecolor{mygray}{RGB}{208,208,208}
\definecolor{mymagenta}{RGB}{226,0,116}
\newcommand*{\mytextstyle}{\sffamily\Large\bfseries\color{black!85}}
\newcommand{\arcarrow}[3]{%
   % inner radius, middle radius, outer radius, start angle,
   % end angle, tip protusion angle, options, text
   \pgfmathsetmacro{\rin}{1.7}
   \pgfmathsetmacro{\rmid}{2.2}
   \pgfmathsetmacro{\rout}{2.7}
   \pgfmathsetmacro{\astart}{#1}
   \pgfmathsetmacro{\aend}{#2}
   \pgfmathsetmacro{\atip}{5}
   \fill[mygray, very thick] (\astart+\atip:\rin)
                         arc (\astart+\atip:\aend:\rin)
      -- (\aend-\atip:\rmid)
      -- (\aend:\rout)   arc (\aend:\astart+\atip:\rout)
      -- (\astart:\rmid) -- cycle;
   \path[
      decoration = {
         text along path,
         text = {|\mytextstyle|#3},
         text align = {align = center},
         raise = -1.0ex
      },
      decorate
   ](\astart+\atip:\rmid) arc (\astart+\atip:\aend+\atip:\rmid);
}
\title[CPP]{Construction Cleanup Copy Move}
\author{Hung Tran}
\institute{Fpt software}
\date{\today}

\begin{document}

\begin{frame}
  \titlepage
\end{frame}

% Uncomment these lines for an automatically generated outline.
\begin{frame}{Outline}
  \tableofcontents
\end{frame}

\section{Constructor Destructor}

\begin{frame}{Constructors}
	\begin{center}
	\textcolor{blue}{\textbf{Default constructor}}
	\textcolor{blue}{\textbf{Parametered constructor}}
	\textcolor{blue}{\textbf{Copy constructor}}
	\end{center}

  \textcolor{blue}{\textbf{A constructor builds a class object "from the bottom up"}}
  \begin{itemize}
    \item The constructor invokes its base class constructor
    \item It invokes the member constructor
    \item It executes its own body
  \end{itemize}

  \textcolor{blue}{\textbf{A destructor tears down a class object "from the bottom up"}}
  \begin{itemize}
    \item It executes its own body
    \item It invokes the member destructor
    \item The constructor invokes its base class destructor
  \end{itemize}
\end{frame}

\begin{frame}{Destructor}
\begin{itemize}
    \item Prevent destructor of an object by declariung its destructor = delete or private
    \item Using private is more flexible (.ex)
    \item Using virtual destructor for a class with virtual function
    \item Need virtual destructor because an object usually manipulated via the interface
    \item Constructor can be virtual because the compiler does not know vtable before object construction
    \item Virtual Constructor idiom is used for c++
  \end{itemize}
\end{frame}

\begin{frame}{Class object initialization}
  \textcolor{blue}{\textbf{Initialization without constructors}}
  \begin{itemize}
      \item Memberwise initialization (if we can access the members)
      \item Copy initialization
      \item Default initialization
  \end{itemize}
  \textcolor{blue}{\textbf{Initialization using constructors}}
  \begin{itemize}
      \item Default constructor disappears when a user constructor defined.
      \item Copy initialization
      \item Default initialization
    \end{itemize}
  \textcolor{blue}{\textbf{Static member initialization}}
  \begin{itemize}
    \item Generally, the static member declaration acts as a declaration for a definition outside the class.
    \item However, it is possible to initialize a static member in class: const of integral or enum types or constexp of literal type.
  \end{itemize}
\end{frame}

\begin{frame}{Copy and move}
  \textcolor{blue}{\textbf{Initialization without constructors}}
  \begin{itemize}
      \item Memberwise initialization (if we can access the members)
      \item Copy initialization
      \item Default initialization
  \end{itemize}
\end{frame}
\end{document}